
\documentclass[a4paper,11pt]{article}  
\pagestyle{plain}
\usepackage{siunitx, dirtytalk, hyperref, nicefrac} 
\usepackage{graphicx,sectsty,longtable,tocloft,color,pdfpages,sidecap,subfig,array,eurosym}
 \captionsetup[figure]{labelfont={bf},name={Fig.},labelsep=period}

%\usepackage{refcheck} % places ??? for unused references in the PDF

\usepackage[english]{babel}
% smaller vertical spacing between references in bibliography
\let\OLDthebibliography\thebibliography
\renewcommand\thebibliography[1]{
  \OLDthebibliography{#1}
  \setlength{\parskip}{0pt}
  \setlength{\itemsep}{0pt plus 0.3ex}
}\usepackage{amsmath}
\usepackage{amsfonts}
\usepackage{amssymb}
%
% 2cm page margins (for A4)
%
\usepackage[left=2cm,right=2cm,top=2cm,bottom=2cm]{geometry}
\tolerance = 1000
\parindent 0cm
\parskip 2mm
%
% Section headings in Arial
%
\allsectionsfont{\sffamily}
%% Customise table of contents behaviour
\tocloftpagestyle{empty}
\renewcommand{\contentsname}{Table of Contents}
\renewcommand{\cfttoctitlefont}{\LARGE\bf\sffamily} 
% Add some width so section label A1.2.3.4 doesn't overlap with section title
\addtolength{\cftsecnumwidth}{3em}
\addtolength{\cftsubsecnumwidth}{3em}
\addtolength{\cftsubsubsecnumwidth}{3em}
%
% Some commonly used definitions
%
\def\urltilda{\kern -.15em\lower .7ex\hbox{\~{}}\kern .04em}
\def\urldot{\kern -.10em.\kern -.10em}
\def\urlhttp{http\kern -.10em\lower -.1ex\hbox{:}\kern -.12em\lower 0ex\hbox{/}\kern -.18em\lower 0ex\hbox{/}}
\def\xilinx{{\small\tt XILINX}}
\def\quarter{{$\frac{1}{4}$}}
\def\half{{$\frac{1}{2}$}}
\def\gm2{{\tt g-2}}
\def\geant{{\tt GEANT}}
\def\art{{\it art}}
\newcommand{\mue[1]}{{\tt Mu{#1}e}}
\newcommand{\gbp[1]}{\pounds\kern 0.08333em{#1}}
\newcommand{\eu[1]}{\euro\kern 0.08333em{#1}}
\newcommand{\usd[1]}{\$\kern 0.08333em{#1}}
%
\begin{document}
\pagenumbering{roman}
\thispagestyle{empty}
\begin{titlepage}
\begin{center}
	%\includegraphics[width=\textwidth]{fig/logo.png}
\end{center}
\begin{center}
	\vspace{1cm}
	{\huge Alignment:}\\
	\vspace{1.5cm}

	\vspace{6cm}
	{\LARGE\bf Gleb Lukicov}\\
	{\Large University College London}
	\vspace{4cm}
	\vfill
	\vspace{0.9cm}
	{\large October 31, 2017}
\end{center}
\end{titlepage}
\clearpage

\pagenumbering{arabic}
\thispagestyle{plain}
\section{Introduction} \label{sec:intro}

PEDE Inputs (x mis.): 


\begin{equation}	
\mathrm{Residual}= R =\mathrm{Point \ To \ Line \ DCA} - \mathrm{Drift \ Circle \ Radius} = DCA(z,x,m,c) - r = \frac{ |c+mz-x| }  { \sqrt{m^2+1} } -r
\end{equation}


\begin{equation}	
\mathrm{Resolution} = \sigma = 150 \ \mu m = 0.015 \ \mathrm{cm}
\end{equation}

\begin{equation}	
\mathrm{Label}= \mathrm{Module \ Number}
\end{equation}

\begin{equation}
DLC_1 = \frac{\partial R}{\partial c} = \frac{ c+mz-x }  { \sqrt{m^2+1} \cdot |c+mz-x| }
\end{equation}

\begin{equation}
DLC_2 = \frac{ \partial R}{\partial m} = \frac{ (m^2+1)\cdot z\cdot(c+mz-x) - m\cdot |c+mz-x|^2 }{ (m^2+1)^{3/2} \cdot |c+mz-x|  }
\end{equation}

\begin{equation}	
DGL_1 = \frac{\partial R}{\partial x} = - DLC_1 = - \frac{ c+mz-x }  { \sqrt{m^2+1} \cdot |c+mz-x| }
\end{equation}



\clearpage
For parallel lines
\begin{equation}
R(x,M) = Track(x) - Hit(x,M) = \hat{m}z+\hat{c} - x(M)	
\end{equation}

\begin{equation}
\frac{ \partial R}{\partial m} = z	= DLC_2
\end{equation}

\begin{equation}
\frac{ \partial R}{\partial c} = 1	=DLC_1 = - DGL_1
\end{equation}

\clearpage

PEDE Inputs for Circle Fit and Misalignment ($M$) along the beam direction ($z=z(M)=z+M$):


\begin{equation}	
\mathrm{Residual}= R =\mathrm{Point \ To \ Line \ DCA} - \mathrm{Drift \ Circle \ Radius} = DCA(z,x,m,c) - r = \frac{ |c+mz-x| }  { \sqrt{m^2+1} } -r
\end{equation}


\begin{equation}	
\mathrm{Resolution} = \sigma = 150 \ \mu m = 0.015 \ \mathrm{cm}
\end{equation}

\begin{equation}	
\mathrm{Label}= \mathrm{Module \ Number}
\end{equation}

\begin{equation}
DLC_1 = \frac{\partial R}{\partial c} = \frac{ c+mz-x }  { \sqrt{m^2+1} \cdot |c+mz-x| }
\end{equation}

\begin{equation}
DLC_2 = \frac{ \partial R}{\partial m} = \frac{ (m^2+1)\cdot z\cdot(c+mz-x) - m\cdot |c+mz-x|^2 }{ (m^2+1)^{3/2} \cdot |c+mz-x|  }
\end{equation}

\begin{equation}	
DGL_1 = \frac{\partial R}{\partial z} = \frac{ m(c+mz-x) }  { \sqrt{m^2+1} \cdot |c+mz-x| },
\end{equation}

 where $DGL_1 \rightarrow \pm \frac{m}{\sqrt{m^2+1}} \rightarrow \pm m$
(using TS expansion: $\pm \frac{m}{\sqrt{m^2+1}} \approx \pm m (1-\frac{1}{2}m^2) = \pm m - \frac{1}{2}m^3 \approx \pm m  $)

\clearpage


For general lines
\begin{equation}	
\mathrm{Residual}= R =\mathrm{Point \ To \ Line \ DCA} - \mathrm{Drift \ Circle \ Radius} = DCA(z,x,m,c) - r = \frac{ |c+mz-x| }  { \sqrt{m^2+1} } -r
\end{equation}


\begin{equation}
DLC_2 = \frac{ \partial R}{\partial m} = \frac{ (m^2+1)\cdot z\cdot(c+mz-x) - m\cdot |c+mz-x|^2 }{ (m^2+1)^{3/2} \cdot |c+mz-x|  } = \frac{\pm z}{\sqrt{m^2+1}} - \frac{m*|c+mz-x|}{(m^2+1)^{3/2}}
\end{equation}

\nocite{*}
\thispagestyle{plain}
%\begin{thebibliography}{100}

\bibitem{mp2} V. Blobel \textit{Software Alignment for Tracking Detectors}, Nucl. Instrum. Methods A, \textbf{556}, 5 (2006).

\bibitem{CMS} F.J.Rongaa et al., \textit{Tracking and Alignment in the CMS Detector}, Nucl. Phys. B, \textbf{172}, 202 (2007).

\bibitem{g-2_TDR} J. Grange et al., \textit{Muon (g-2) Technical Design Report}, arXiv:1501.06858 (2015).

\bibitem{art} A. Lyon et al., \textit{The art Event-Processing Framework}, \url{https://web.fnal.gov/project/ArtDoc/Pages/home.aspx} (2015).

\bibitem{Tom} T. Stuttard, \textit{The Development, Testing and Characterisation of a Straw Tracking Detector and Readout System for the Fermilab Muon g-2 Experiment}, PhD thesis, University College London (2017).

\bibitem{GBL} C. Kleinwort, \textit{General Broken Lines as Advanced Track Fitting Method}, Nucl. Instrum. Methods A \textbf{673}, 107 (2012).

\bibitem{GARFIELD} R. Veenhof, \textit{Garfield: Simulation of Gaseous Detectors}, \url{http://garfield.web. cern.ch/garfield} (2010).


\end{thebibliography}

%%%%% CLEAR DOUBLE PAGE!
\newpage{\pagestyle{empty}\cleardoublepage}
\addcontentsline{toc}{chapter}{\numberline{}\sf\bfseries{References}}
\end{document}