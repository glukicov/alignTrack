\documentclass[a4paper,11pt]{article}  
\pagestyle{plain}
\usepackage{siunitx, dirtytalk, hyperref, nicefrac, mathtools} 
\usepackage{graphicx,sectsty,longtable,tocloft,color,pdfpages,sidecap,subfig,array,eurosym}
 \captionsetup[figure]{labelfont={bf},name={Fig.},labelsep=period}

%\usepackage{refcheck} % places ??? for unused references in the PDF

\usepackage[english]{babel}
% smaller vertical spacing between references in bibliography
\let\OLDthebibliography\thebibliography
\renewcommand\thebibliography[1]{
  \OLDthebibliography{#1}
  \setlength{\parskip}{0pt}
  \setlength{\itemsep}{0pt plus 0.3ex}
}\usepackage{amsmath}
\usepackage{amsfonts}
\usepackage{amssymb}
%
% 2cm page margins (for A4)
%
\usepackage[left=2cm,right=2cm,top=2cm,bottom=2cm]{geometry}
\tolerance = 1000
\parindent 0cm
\parskip 2mm
%
% Section headings in Arial
%
\allsectionsfont{\sffamily}
%% Customise table of contents behaviour
\tocloftpagestyle{empty}
\renewcommand{\contentsname}{Table of Contents}
\renewcommand{\cfttoctitlefont}{\LARGE\bf\sffamily} 
% Add some width so section label A1.2.3.4 doesn't overlap with section title
\addtolength{\cftsecnumwidth}{3em}
\addtolength{\cftsubsecnumwidth}{3em}
\addtolength{\cftsubsubsecnumwidth}{3em}
%
% Some commonly used definitions
%
\def\urltilda{\kern -.15em\lower .7ex\hbox{\~{}}\kern .04em}
\def\urldot{\kern -.10em.\kern -.10em}
\def\urlhttp{http\kern -.10em\lower -.1ex\hbox{:}\kern -.12em\lower 0ex\hbox{/}\kern -.18em\lower 0ex\hbox{/}}
\def\xilinx{{\small\tt XILINX}}
\def\quarter{{$\frac{1}{4}$}}
\def\half{{$\frac{1}{2}$}}
\def\gm2{{\tt g-2}}
\def\geant{{\tt GEANT}}
\def\art{{\it art}}
\newcommand{\mue[1]}{{\tt Mu{#1}e}}
\newcommand{\gbp[1]}{\pounds\kern 0.08333em{#1}}
\newcommand{\eu[1]}{\euro\kern 0.08333em{#1}}
\newcommand{\usd[1]}{\$\kern 0.08333em{#1}}
%
\begin{document}
\pagenumbering{roman}
\thispagestyle{empty}
\begin{titlepage}
\begin{center}
    %\includegraphics[width=\textwidth]{fig/logo.png}
\end{center}
\begin{center}
    \vspace{1cm}
    {\huge \textbf{Alignment Uncertainty:} \\ \textit{Estimating the Contribution of the Alignment to the Beam Extrapolation and the CBO} }\\
    \vspace{1.5cm}
    {\Huge \textbf{ DRAFT }}\\
    \vspace{6cm}
    {\LARGE\bf Gleb Lukicov}\\
    {\Large University College London}
    \vspace{4cm}
    \vfill
    \vspace{0.9cm}
    {\large January 18, 2019}
\end{center}
\end{titlepage}
\clearpage

\pagenumbering{arabic}
\thispagestyle{plain}

\clearpage
\section{Introduction}

In order for the tracking detector to reduce the systematic uncertainty on the $a_{\mu}$ measurement and improve the sensitivity to a muon EDM, the absolute position of the tracking modules must be known to a high level of precision. Individual straw effects such as reduced wire tension, can affect different straws in a different way. Therefore, a physics-level (i.e.~track-based) alignment, that considers such effects, is required. Track-based alignment is implemented with data from Run 1 using the \texttt{Millepede II} framework~\cite{mp2}. A Monte Carlo (MC) simulation was developed to understand the detector geometry and how this affects how well the alignment can be determined, as well as to test the alignment procedure itself. The beam extrapolation (as shown in Fig.~\ref{fig:beamSpot}.b) will also greatly benefit from the internal alignment of the tracker.
\begin{figure}[ht!]
    \centering
    \includegraphics[height=6 cm]{fig/beamSpot}
    \vspace{-8pt}
    \caption{Reconstructed radial and vertical beam position from tracks that have been extrapolated back to their decay position.}
    \label{fig:beamSpot}
\end{figure}

It is also imperative to have an estimate of the systematic uncertainty that comes from an un-aligned detector. One way to produce such an estimate is to add sets of known misalignment offsets to the detector and reconstruct data with these offsets. After reconstruction is done, a comparison can be made between the nominal case and the case with the added offsets. It is important to note, that the nominal case itself has some real and unknown misalignment. Moreover, an analytical estimation of the misalignment contrition to the track extrapolation is performed. 


\section{Methodology}  

The focus will be on the contribution of the added offsets to the quality of the extrapolated beam, namely its RMS and the mean. Other physics measurements, such as the CBO, will also be used for the estimation of the alignment error. Here, four misalignment scales will be mapped out (\SI{25}{\micro\metre}, \SI{50}{\micro\metre}, \SI{100}{\micro\metre}, and \SI{200}{\micro\metre}), each additionally with 9 set points of the overall mean shift in the modules (\SI{0}{\micro\metre}, $\pm$\SI{25}{\micro\metre},  $\pm$\SI{50}{\micro\metre},  $\pm$\SI{100}{\micro\metre}, and $\pm$\SI{200}{\micro\metre}). These scales of misalignment are comparable to what have been seen with data (see next chapter). Each mapping set of offsets will have 25 cases of random offsets drawn from a uniform distribution. This corresponds to reconstruction (tracking only) of equivalent of 900 runs of the g-2 experiment worth of data (92 TB). Such a monumental task will be performed by most efficient utilisation of the distributed grid computing resources of the Open Science Grid \cite{OSG}.

A chosen run (nominal case) for this study was Run 15922 (22 April 2018) that contains 1 hour worth of physics-quality data, and the extrapolated radial and vertical beam position for both stations is shown in Fig.~\ref{fig:nominalBeam}. All extrapolation is done with p-value $>$ 0.005 and requiring that the tracks did not hit a volume (e.g. a vacuum chamber) before producing a track in the detector, to select the best quality tracks for the study. The plots have further been cut on tails between $\pm$\SI{70}{\milli\metre}, to only look at the tracks that have come from a uniform field region.


\begin{figure}[ht!]
    \centering
    \includegraphics[height=8 cm]{fig/nominalBeam}
    \vspace{-8pt}
    \caption{Run 15922. Nominal radial and vertical beam position.}
    \label{fig:nominalBeam}
\end{figure}





\clearpage


\nocite{*}
\thispagestyle{plain}
\begin{thebibliography}{100}

\bibitem{mp2} V. Blobel \textit{Software Alignment for Tracking Detectors}, Nucl. Instrum. Methods A, \textbf{556}, 5 (2006).   

\end{thebibliography}

%%%%% CLEAR DOUBLE PAGE!
\newpage{\pagestyle{empty}\cleardoublepage}
\addcontentsline{toc}{chapter}{\numberline{}\sf\bfseries{References}}
\end{document}